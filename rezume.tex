%-------------------------
% Rezume, a latex resume template for developers
% Author : Nanu Panchamurthy
% Based off of: https://github.com/sb2nov/resume
% License : MIT

% Hope this resume template helps you land an awesome job. If you found this helpful, please consider starring the github repo here, .
%-------------------------



%------------PACKAGES----------------
\documentclass[a4paper,11pt]{article}

\usepackage{verbatim} % reimplements the "verbatim" and "verbatim*" environments

\usepackage{titlesec} % provides an interface to sectioning commands i.e. custom elements

\usepackage{color} % provides both foreground and background color management

\usepackage{enumitem} % provides control over enumerate, itemize and description

\usepackage{fancyhdr} % provides extensive facilities for constructing headers, footers and also controlling their use

\usepackage{tabularx} % defines an environment tabularx, extension of "tabular" with an extra designator x, paragraph like column whose width automatically expands to fill the width of the environment

\usepackage{latexsym} % provides mathematical symbols

\usepackage{marvosym} % provides martin vogel's symbol font which contains various symbols

\usepackage[empty]{fullpage} % sets margins to one inch and removes headers, footers etc..

\usepackage[hidelinks]{hyperref} % removes color and shadow of hyperlinks

\usepackage[normalem]{ulem} % provides "\ul" (uline) command which will break at line breaks

\usepackage[english]{babel} % provides culturally determined typographical rules for wide range of languages
%-----------------------------------------

\input glyphtounicode % converts glyph names to unicode
\pdfgentounicode=1 % ensures pdfs generated are ats readable

%----------FONT OPTIONS-------------------
\usepackage[default]{sourcesanspro} % uses the font source sans pro
\urlstyle{same} % changes url font from default urlfont to font being used by the document
%-----------------------------------------


%----------MARGIN OPTIONS-----------------
\pagestyle{fancy} % set page style to one configured by fancyhdr
\fancyhf{} % clear all header and footer fields

\renewcommand{\headrulewidth}{0in} % sets thickness of linerule under header to zero
\renewcommand{\footrulewidth}{0in} % sets thickness of linerule over footer to zero

\setlength{\tabcolsep}{0in} % sets thickness of column separator in tables to zero

% origin of the document is one inch from the top and from and the left
% oddsidemargin and evensidemargin both refer to the left margin
% right margin is indirectly set using oddsidemargin
\addtolength{\oddsidemargin}{-0.5in}
\addtolength{\topmargin}{-0.5in}

\addtolength{\textwidth}{1.0in} % sets width of text area in the page to one inch
\addtolength{\textheight}{1.0in} % sets height of text area in the page to one inch

\raggedbottom{} % makes all pages the height of current page, no extra vertical space added
\raggedright{} % makes all pages the width of current page, no extra horizontal space added
%------------------------------------------


%--------SECTIONING COMMANDS---------
% \titleformat{<command>}
%   [<shape>]{<format>}{<label>}{<sep>}
%     {<before-code>}[<after-code>]

% command is the sectioning command to be redefined
% shape is the style of the font; scshape stands for small caps style
% format is the format to be applied to whole title- label and text; absent here
% label defines the label
% sep is the horizontal separation between label and title body
% before-code is the code to be executed before
% after-code is the code to be executed after

\titleformat{\section}
  {\scshape\large}{}
    {0em}{\color{blue}}[\color{black}\titlerule\vspace{0pt}]
%-------------------------------------


%--------REDEFINITIONS----------------
% redefines the style of the bullet point
\renewcommand\labelitemii{$\vcenter{\hbox{\tiny$\bullet$}}$}

% redefines the underline depth to 2pt
\renewcommand{\ULdepth}{2pt}
%-------------------------------------


%--------CUSTOM COMMANDS--------------
%\vspace{} defines a vertical space of given size, modifying this in custom commands can help stretch or shrink resume to remove or add content

% resumeItem renders a bullet point
\newcommand{\resumeItem}[1]{
  \item\small{#1}
}

% commands to start and end itemization of resumeItem, rightmargin set to 0.11in to avoid the overflow of resumetItem beyond whatever resumeItemHeading is being used
\newcommand{\resumeItemListStart}{\begin{itemize}[rightmargin=0.11in]}
\newcommand{\resumeItemListEnd}{\end{itemize}}

% resumeSectionType renders a bolded type to be used under a section, used as skill type here, middle element is used to keep ":"s in the same vertical line
\newcommand{\resumeSectionType}[3]{
  \item\begin{tabular*}{0.66\textwidth}[t]{
    p{0.15\linewidth}p{0.02\linewidth}p{0.81\linewidth}
  }
    \textbf{#1} & #2 & #3
  \end{tabular*}\vspace{-2pt}
}

% resumeTrioHeading renders three elements in three columns with second element being italicized and first element bolded, can be used for projects with three elements
\newcommand{\resumeTrioHeading}[3]{
  \item\small{
    \begin{tabular*}{0.96\textwidth}[t]{
      l@{\extracolsep{\fill}}c@{\extracolsep{\fill}}r
    }
      \textbf{#1} & \textit{#2} & #3
    \end{tabular*}
  }
}


% resumeQuadHeading renders four elements in a two columns with the second row being italicized and first element of first row bolded, can be used for experience and projects with four elements
\newcommand{\resumeQuadHeading}[4]{
  \item
  \begin{tabular*}{0.96\textwidth}[t]{l@{\extracolsep{\fill}}r}
    \textbf{#1} & #2 \\
    \textit{\small#3} & \textit{\small #4} \\
  \end{tabular*}
}

% resumeQuadHeadingChild renders the second row of resumeQuadHeading, can be used for experience if different roles in the same company need to added
\newcommand{\resumeQuadHeadingChild}[2]{
  \item
  \begin{tabular*}{0.96\textwidth}[t]{l@{\extracolsep{\fill}}r}
    \textbf{\small#1} & {\small#2} \\
  \end{tabular*}
}

% commands to start and end itemization of resumeQuadHeading, lefmargin for left indent of 0.15in for resumeItems
\newcommand{\resumeHeadingListStart}{
  \begin{itemize}[leftmargin=0.15in, label={}]
}
\newcommand{\resumeHeadingListEnd}{\end{itemize}}
%-------------------------------------------


%__________________RESUME____________________
% You can rearrange sections in any order you may prefer
\begin{document}

%-----------CONTACT DETAILS------------------
% Make sure all the details are correct, you can add more links in the first row of second column if needed

\begin{tabular*}{\textwidth}{l@{\extracolsep{\fill}}r}
  \textbf{\Huge Yash Patil\vspace{2pt}} & % row = 1, col = 1
  Location: Ulhasnagar, Maharashtra, India \\ % row = 1, col = 2
  \href{https://frost2k5.is-a.dev/}{\uline{Personal Website}} $|$ % row = 2, col = 1
  \href{https://www.linkedin.com/in/yash-patil-385171257/}{\uline{LinkedIn}} $|$ % row = 2, col = 1
  \href{https://github.com/FrosT2k5}{\uline{GitHub}} $|$ % row = 2, col = 1
  Email: \href{mailto:yashpatil2k5@gmail.com}{\uline{yashpatil2k5@gmail.com}} $|$ % row = 2, col = 2
\end{tabular*}
%--------------------------------------------


%-----------SUMMARY--------------------------
% Keep this short, simple and straigth to point

% \section{Summary}
% \small{
%   \textbf{Self-driven technologist} with hands-on experience in \textbf{full-stack development}, \textbf{cybersecurity}, and \textbf{reverse engineering}. Actively building and deploying systems across \textbf{Linux}, \textbf{Android}, and \textbf{web platforms}, with a deep interest in \textbf{software architecture}, \textbf{system security}, and \textbf{low-level tooling}.
% }
%--------------------------------------------

%-----------EDUCATION-------------------------
% Mention your CGPA, if its good, in the first row of second column

\section{Education}
  \resumeHeadingListStart{}
    \resumeQuadHeading{University of Mumbai}{K. C. College, Thane}
    {\textbf{B.E.} in \textbf{Information Technology} with Honours in Cyber Security} {Nov 2022 -- June 2026}
    {CGPA (till sem 6): 9.09; IETE Student Forum Best Member '24-'25}
  \resumeHeadingListEnd{}
  % \resumeHeadingListStart{}
  %   \resumeQuadHeading{HSC - Maharashtra Board}{RKT College, Ulhasnagar}
  %   {Grade: 62\%}{Apr 2022}
  % \resumeHeadingListEnd{}
%---------------------------------------------

%-----------EXPERIENCE-----------------------
% Distill all your talking points to small bullet points which follow the pattern "challenge-action-result" for maximum efficiency. Try to quantify (use numbers) your points whenver possible, highlist words of importance

\section{Experience}
\resumeHeadingListStart{}
  \resumeQuadHeading{Cyber Security Research Intern: Android Internals \& Reverse Engineering}{Feb 2025 -- Present}
  {Center of Excellence (CoE) in Complex and NonLinear Dynamic Systems (CNDS) - \textbf{VJTI}}{}
    \resumeItemListStart{}
      \resumeItem{Worked with the researcher team to \textbf{enhance internal tooling} for reverse engineering and analysis of the \textbf{baseband processor} on Android}
      \resumeItem{Recreated \textbf{CVE-2021-25479} and \textbf{CVE-2021-25478} using the open-source \textbf{FirmWire} framework in an emulated environment, following documented workflows involving \textbf{fuzzing}, \textbf{system emulation}, and \textbf{static analysis} with Ghidra and IDA}
      \resumeItem{Developed a local cellular network test network using \textbf{SDR} and open-source stacks like srsRAN, OpenLTE, OpenBTS, YatesBTS, and OpenAirInterface}
      \resumeItem{Built APK static analysis scripts to automate reverse engineering workflows using \textbf{JadX} and \textbf{Ghidra}}
      \resumeItem{Researched the fundamentals of Android’s \textbf{Binder IPC} mechanism and its associated vulnerabilities}
    \resumeItemListEnd{}
\resumeHeadingListEnd{}
%---------------------------------------------

%-----------PROJECTS--------------------------
% Use resumeQuadHeading if four elements are feasible (ex: demo video link), else use resumeTrioHeading. Keep the bullet points simple and concise and try to cover wide variety of skills you have used to build these projects

\section{Projects}
  \resumeHeadingListStart{}

    \resumeTrioHeading{\href{https://frost.alwaysdata.net/}{\uline{AirQualitySensor}}}{ESP32 (C/C++), React (TS), Firebase, aWOT, ArduinoJson}{\href{https://github.com/FrosT2k5/AirQualitySensor}{\uline{Source Code}}}
    \resumeItemListStart{}
      \resumeItem{Developed firmware with \textbf{ESP-IDF RTOS} in modular \textbf{C/C++}, implementing \textbf{watchdog handling}, \textbf{multicore task scheduling}, and task pinning with \textbf{aWOT} and \textbf{ArduinoJson} for REST APIs.}
      \resumeItem{Built a \textbf{React + TypeScript SPA} with Router and Firebase Realtime DB integration for real-time visualization.}
      \resumeItem{Enabled dual connectivity (LAN + Cloud) by syncing ESP and frontend with Firebase alongside local APIs.}
    \resumeItemListEnd{}

    
    \resumeTrioHeading{\href{https://datadash.is-a.dev/}{\uline{Datadash}}}{Java, Python, C/C++, cryptography, GitHub CI/CD}{\href{https://github.com/Armaan4477/Datadash}{\uline{Source Code}}}
    \resumeItemListStart{}
      \resumeItem{Built a secure file transfer system with \textbf{AES-256 encryption} across \textbf{Android, Linux, Windows, and Python}.}
      \resumeItem{Implemented \textbf{socket-level programming} for direct file transfer with crash recovery and preference management.}
      \resumeItem{Created a \textbf{GitHub CI/CD pipeline} to automate compilation of Linux binaries and Windows executables; also contributed to the landing page backend.}
    \resumeItemListEnd{}

        
      
        {\textbf{Other Projects}}{:}
        \resumeItemListStart{}
       
            \resumeItem{\href{https://openmedia-e0gvb8fzbnexafd8.centralus-01.azurewebsites.net/}{\uline{OpenMedia}} – Flask, SQLAlchemy, Bootstrap; microblogging platform with media uploads and privacy features.}
            \resumeItem{\href{https://github.com/FrosT2k5/diary}{\uline{Digital Encrypted Diary}} – GnuPG + Python/Bash; CLI diary with Git-based sync and year-wise encrypted files.}
            \resumeItem{\href{https://github.com/NandiniNichite/book-recommender-system}{\uline{ReadNexus}} – Py/Flask + SBERT + Pandas + Bootstrap; full-stack book recommender with semantic search, collaborative filtering, and popularity-based suggestions.}

        
    
        \resumeItemListEnd{}
      
  \resumeHeadingListEnd{}
%--------------------------------------------


%--------------SKILLS------------------------
% Add or remove resumeSectionTypes according to your needs

\section{Technical Skills}
  \resumeHeadingListStart{}
    \resumeSectionType{Languages}{:}{Python, JavaScript/TypeScript, Java, C/C++, Bash}
    \resumeSectionType{Frameworks}{:}{Django, Flask, React, Node.js, Express, SQLAlchemy, Mongoose, Bootstrap, REST APIs, JWT}
    \resumeSectionType{Tools}{:}{Git, Docker, Azure, Postman, CLI, CI/CD, Swagger/OpenAPI, GNU/Linux utilities}
  \resumeHeadingListEnd{}

%--------------------------------------------


%----------------OTHERS----------------------
% You can add your acheivements, accolades, certifications etc. here.

\section{Certifications and Achievements}
\resumeItemListStart{}
  \resumeItem{1\textsuperscript{st} – Debugging (Technotia, Kamaladevi) \& Codesprint Codathon (GDSC, KC@2023) ; Dept. Topper (IT, 2023–24)} 
  \resumeItem{2\textsuperscript{nd} – Reverse Coding (Technotia) ; 2\textsuperscript{nd} – T915 Android Internals CTF (VJTI) ; 3\textsuperscript{rd} – SIH Internal Hackathon (K.C. College)}
  \resumeItem{Virtual Internships via Eduskills: \href{https://aictecert.eduskillsfoundation.org/pages/home/verify.php?cert=302f1886823167c7ef0ef578223c072f}{\uline{CyberSecurity}}, \href{https://aictecert.eduskillsfoundation.org/pages/home/verify.php?cert=9d7218a4a5c5ededd2b49dfc987bbf61}{\uline{Android Developer}}, \href{https://aictecert.eduskillsfoundation.org/pages/home/verify.php?cert=13bcb2d0ecf9d2b4c0cfd2e236300456}{\uline{Data Science Master}}, \href{https://aictecert.eduskillsfoundation.org/pages/home/verify.php?cert=cfae1e9e1eecb2a6b907475e90b5e31f}{\uline{Computer Networks}}}

\resumeItemListEnd{}

%--------------------------------------------

\end{document}
